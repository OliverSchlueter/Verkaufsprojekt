\chapter{Umsetzung}

In diesen Kapitel werden alle Klassen mit ihren Attributen von Methoden/Funktionen beschrieben.

\section{Program.cs}
In dieser Klasse wird das Programm gestartet. Außerdem enthält sie Informationen über die Version des Programms. Um immer wieder zurück zur Startseite zu gelangen befindet sich in der Program.cs Klasse das statische StartseiteForm Objekt, wobei hier das Singleton Entwurfsmuster angewandt wird, d.h. es soll nur eine Instanz der Klasse StartseiteForm existieren. Ebenfalls enthält sie Informationen über den gerade angemeldeten Benutzer (ebenfalls mit den Singleton Entwurfsmustern angewandt).

\section{Benutzer.cs}

Diese Klasse speichert alle Informationen über einen Benutzer.\\
\\
Attribute:\\

\begin{tabular}[h]{l|l|l|p{8cm}}
Modifizierer & Typ & Name & Beschreibung\\
\hline
public static & List<Benutzer> & BENUTZER & Liste alle registrierten Benutzer in der Datenbank\\
\hline
public & string & BenutzerID & Eindeutige ID, immer 5 Zeichen lang\\
\hline
public & string & Vorname & Vorname des Benutzers\\
\hline
public & string & Vorname & Nachname des Benutzers\\
\hline
public & string & Nickname & Nickname des Benutzers\\
\hline
public & string & Email & Email-Adresse des Benutzers\\
\hline
public & DateTime & Geburtsdatum & Geburtsdatum des Benutzers\\
\hline
public & DateTime & Erstelldatum & Erstelldatum, wann der Benutzer erstellt wurde\\
\hline
public & string & Passwort & Passwort des Benutzers (mindestens 8 Zeichen lang)\\
\end{tabular}

\newpage
Methoden/Funktionen:\\

\begin{tabular}[h]{l|l|l|p{7cm}}
Modifizierer & Typ & Name & Beschreibung\\
\hline
public static & string & GenerateNewBenutzerID() & Generiert eine nicht schon vorhandene neue Benutzer ID\\
\hline
public static & Benutzer & getBenutzerFromID(string id) & Gibt das Benutzer Objekt mit der zugehörigen ID zurück. Null wenn nicht gefunden\\
\hline
public static & Benutzer & LoadFromDB() & Lädt alle Benutzer von der Datenbank in die BENUTZER-Liste\\
\end{tabular}


\subsection{Kunde.cs}
Kunde.cs ist eine Klasse, welche von Benutzer.cs erbt.\\
\\
Attribute:\\

\begin{tabular}[h]{l|l|l|p{8cm}}
Modifizierer & Typ & Name & Beschreibung\\
\hline
public static & List<Kunde> & KUNDEN & Liste alle registrierten Kunden in der Datenbank\\
\hline
public & float & Guthaben & Guthaben des Kunden\\
\hline
public & List<Produkt> & GekaufteProdukte & Liste der bereits gekauften Produkte\\
\hline
public & List<Produkt> & Beobachtungsliste & Liste der Produkte, welche der Kunde beobachtet\\
\hline
public & List<Produkt> & Wunschliste & Liste der Produkte, welche der Kunde sich wünscht\\
\end{tabular}\newline \break

Methoden/Funktionen:\\

\begin{tabular}[h]{l|l|p{6cm}|p{6cm}}
Modifizierer & Typ & Name & Beschreibung\\
\hline
public & bool & produktKaufen(Produkt produkt) & Zieht Guthaben ab und weist das Produkt den Kunde zu\\
\hline
public & bool & produktWuenschen(Produkt produkt) & Fügt das Produkt in die Wunschliste hinzu\\
\hline
public & bool & produktBeobachten(Produkt produkt) & Fügt das Produkt in der Beobachtungsliste hinzu\\
\hline
public & bool & produktBewerten(Produkt produkt, byte sterne, string kommentar) & Fügt eine Bewertung zum Produkt zu\\
\hline
public static & Benutzer & geKundeFromID(string id) & Gibt das Kunden Objekt mit der zugehörigen ID zurück. Null wenn nicht gefunden\\
\hline
public static & Benutzer & LoadFromDB() & Lädt alle Kunden von der Datenbank in die KUNDEN-Liste\\
\end{tabular}

\newpage

\subsection{Autor.cs}
Autor.cs ist eine Klasse, welche von Benutzer.cs erbt.\\
\\
Attribute:\\

\begin{tabular}[h]{l|l|l|p{5.3cm}}
Modifizierer & Typ & Name & Beschreibung\\
\hline
public static & List<Autor> & AUTOREN & Liste alle registrierten Autoren in der Datenbank\\
\hline
public & Entwicklerstatus & Entwicklerstatus & Beschreibt den Entwicklerstatus des Autors\\
\hline
public & List<Programmiersprache> & Programmiersprachen & Liste der Programmiersprachen, welche der Autor beherrscht\\
\hline
public & List<Produkt> & Beobachtungsliste & Liste der Produkte, welche der Kunde beobachtet\\
\hline
public & List<Produkt> & Wunschliste & Liste der Produkte, welche der Kunde sich wünscht\\
\end{tabular}\newline \break

Methoden/Funktionen:\\

\begin{tabular}[h]{l|l|p{6cm}|p{6cm}}
Modifizierer & Typ & Name & Beschreibung\\
\hline
public static & Benutzer & LoadFromDB() & Lädt alle Autoren von der Datenbank in die AUTOREN-Liste\\
\end{tabular}


\subsection{Mitarbeiter.cs}
Mitarbeiter.cs ist eine Klasse, welche von Benutzer.cs erbt.\\
\\
Attribute:\\

\begin{tabular}[h]{l|l|l|p{8cm}}
Modifizierer & Typ & Name & Beschreibung\\
\hline
public static & List<Mitarbeiter> & MITARBEITER & Liste alle registrierten Mitarbeiter in der Datenbank\\
\hline
public & Aufgabenbereich & Aufgabenbereich & Gibt den Aufgabenbereich des Mitarbeiters an\\
\end{tabular}\newline \break

\newpage

\section{Produkt.cs}
In dieser Klasse werden Informationen über ein Produkt gespeichert.\\
\\

Attribute:\\

\begin{tabular}[h]{l|l|l|p{6.3cm}}
Modifizierer & Typ & Name & Beschreibung\\
\hline
public static & List<Produkt> & PRODUKTE & Liste alle registrierten Produkte in der Datenbank\\
\hline
public & string & ID & Eindeutige ID\\
\hline
public & string & Name & Name des Produkts\\
\hline
public & List<Autor> & Autoren & Liste aller Autoren, welche das Produkt geschrieben haben\\
\hline
public & float & Preis & Preis des Produkts (0 wenn es kostenfrei ist)\\
\hline
public & string & Beschreibung & Ausführliche Beschreibung des Produkts\\
\hline
public & DateTime & Veröffentlichungsdatum & Datum der Veröffentlichung des Produkts\\
\hline
public & List<Bewertung> & Bewertungen & Liste aller Bewertungen von Kunden\\
\end{tabular}\\\\

Methoden/Funktionen:\\

\begin{tabular}[h]{l|l|l|p{7cm}}
Modifizierer & Typ & Name & Beschreibung\\
\hline
public static & Benutzer & getProduktFromID(string id) & Gibt das Produkt Objekt mit der zugehörigen ID zurück. Null wenn nicht gefunden\\
\hline
public static & Benutzer & LoadFromDB() & Lädt alle Produkte von der Datenbank in die PRODUKTE-Liste\\
\end{tabular}\\\\

\section{Bewertung.cs}
In dieser Klasse werden Informationen über eine Bewertung gespeichert.\\
\\

Attribute:\\

\begin{tabular}[h]{l|l|l|p{8cm}}
Modifizierer & Typ & Name & Beschreibung\\
\hline
public & Kunde & Kunde & Kunde, welcher die Bewertung abgegeben hat\\
\hline
public & byte & Sterne & Anzahl der gegebene Sterne (1-5)\\
\hline
public & string & Kommentar & Kommentar zum Produkt\\
\hline
public & boolean & VerifizierterKauf & Ob das Produkt gekauft wurde oder nicht\\
\end{tabular}

\newpage

Methoden/Funktionen:\\

\begin{tabular}[h]{l|l|l|p{7cm}}
Modifizierer & Typ & Name & Beschreibung\\
\hline
public static & Benutzer & getProduktFromID(string id) & Gibt das Produkt Objekt mit der zugehörigen ID zurück. Null wenn nicht gefunden\\
\hline
public static & Benutzer & LoadFromDB() & Lädt alle Produkte von der Datenbank in die PRODUKTE-Liste\\
\end{tabular}\\\\

\section{DatabaseManager.cs}
Dies ist eine Hilfsklasse, welche den Zugriff zur Datenbank vereinfacht.\\
\\

Attribute:\\

\begin{tabular}[h]{l|l|l|p{7cm}}
Modifizierer & Typ & Name & Beschreibung\\
\hline
public static & DatebaseManager & Database & Globales Database Objekt\\
\hline
private & string & ConnectionString & Connection-String, welcher zur Verbindung benötigt wird\\
\hline
public & OleDbConnection & Connection & Verbindungs-Objekt zur Datenbank\\
\hline
public & OleDbCommand & Command & Command-Objekt zur Ausführung von Befehlen\\
\hline
public & OleDbDataReader & Reader & Reader-Objekt zur Auslesung von Resultaten\\
\end{tabular}

Methoden/Funktionen:\\

\begin{tabular}[h]{l|l|l|p{5cm}}
Modifizierer & Typ & Name & Beschreibung\\
\hline
public & void & execute(string sql) & Führt ein Befehl aus. Geeignet für INSERT-Befehle\\
\hline
public & OleDbDataReader & Read(string sql) &  Führt ein Befehl aus und gibt den Reader zurück\\
\hline
public & void & CloseReaderAndConnection() & Schließt die Verbindung zum Reader und zur Connetion\\
\hline
public & List<object[]> & GetData(string sql) & Führt ein Befehl aus und liefert eine Liste (Reihen) mit Arrays (Spalten) von Daten\\
\hline
private static & void & example() & Beispiel, nicht benutzen\\
\end{tabular}












